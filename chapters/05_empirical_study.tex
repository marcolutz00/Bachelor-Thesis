\chapter{Experiment}\label{chapter:Experiment}

\section{Experiment Setup}
In this thesis, we test the capabilities of LLMs to generate accessible code in an 
Image-to-Code environment. Therefore, a pipeline is used which provides the LLMs an 
image with instructions to replicate this image in HTML/CSS. In the following, 
advanced prompting techniques are also provided to guide the LLMs to generate
accessible code. 
The output of the LLMs for each parameter tuple (Model, Prompting Technique) 
is then analyzed on visual and structural 
similarity, as well as on accessibility compliance, based on WCAG 2.2 standards.
In order to cope with the probabilistic nature of LLMs, each experiment with a 
predefined tuple of parameters will be tested within 3 experiment runs.\newline
After the generation of the LLMs' output for the Image-to-Code task, the HTML/CSS
is first analyzed on visual and structural similarity as described above and afterwards
analyzed on accessibility violations with the help of axe-core, lighthouse and pa11y.


\subsection{Model Selection}
The selection of the LLMs is a decisive factor for the interpretation of the results.
To see the diffences of performance, we decided to use different types of 
state-of-the-art models. They differentiate in size, architecture and use case, but 
they all have vision capabilities in common.
In order to provide a general picture, we identified three model groups of interest:
\begin{itemize}
  \item \textbf{Commercial:} Big, commercial models build the foundation of our tests.
    We use \textit{gpt-4o} and \textit{gemini flash 2.0} as representatives of this group.
    They have not been specifically trained for Image-to-Code tasks, but prior papers 
    already show promising results in this field.
  \item \textbf{Small, Open-Source:} Small, open-source models build the second group. 
    Theoretically, they should be more accessible for the public and could be hosted 
    on local machines. The representatives of this group are \textit{llama 3.2 vision}
    and \textit{qwen 7B vl}. While the models are signifant smaller, they still have been 
    trained on a lot of data.
  \item \textbf{Fine-tuned:} The last group are fine-tuned models. In this case, we 
    consider models which have been trained specifically on similar Image-to-Code 
    tasks. We use \textit{VLM WebSight finetuned} which is a fine-tuned version official
    the models \textit{SigLIP} and \textit{Mistral 7B v0.1}. It has been trained on the 
    \textit{Websight} dataset which contains more than 800,000 HTML/CSS data entries. 
\end{itemize}

\section{Prompting Techniques}
In order to understand the models' capabilities to generate accessible code, we test
the models with different prompting techniques. Prior research has shown that more 
advanced prompting techniques can help to improve the generation of robust and 
accurate code. Therefore, we combine commonly used techniques with 
advanced prompting techniques to guide the LLMs to create more accessible code.
The wording of the prompts is similar to the one of prior research~\parencite{suh2025accessiblecode, xiao2024interaction2code}.
The content is the same while we have enriched the prompts with more details.\newline
All prompts can be seen in the appendix.

\subsection{Naive}
The naive approach only instructs the LLMs to accurately fulfill Image-to-Code tasks
without mentioning accessibility in its prompt. This shows how state-of-the-art 
LLMs perform in generating accessible code naturally without instructing it to 
do so.

\subsection{Zero-Shot}
The zero-shot approach resembles the naive approach in terms of the Image-to-Code
instructions. However, here the LLMs are explicitly instructed to obey the WCAG 
2.2 standards. The prompt does not contain examples, however it emphasizes 
accessibility by reminding it about the WCAG standards including a link to 
official WCAG standards.

\subsection{Few-Shot}
The few-shot approach resembles the zero-shot prompt, however it is enriched with 
explicit examples to support the LLM to generate more accessible code.
Since the number of possible examples is too large, we decided to 

\subsection{Reasoning}
The reasoning prompt is used to let the LLMs reason about the task and possible 
accessibility problems within the generation. Similar to prior work, we use 
a prompt to generate reasoning steps with \"Let’s think step by 
step.\" as part of the output~\parencite{chae2024thinkexecute}.

\subsection{Iterative}
The iterative approach follows a similar approach to recent findings in the area
of generating more accessible code~\parencite{suh2025accessiblecode}. This 
prompting technique lets the LLMs generate Image-to-Code tasks by using the naive 
prompting technique in the base iteration. Afterwards, the generated output is 
analyzed and accessibility violations found are incorporated with a refinement 
message to the LLMs. The objective is to instruct the LLMs to create a more 
accessible code based on the violations found in the code. The iterative approach 
contains one naive prompt and three rounds of code refinement. If the LLMs 
generate code without any accessibility violations within one round, the 
iterations stop.


\section{Accessibility Evaluation}
In order to provide a comprehensive analysis of the accessibility violations, 
we divide the analysis into 3 categories. Those categories differentiate in 
terms of quantitive and qualitive analysis as well as in its depth.

\subsection{Quantitive Analysis}
Figure x shows the quantitive analysis of the amount and type of Accessibility 
Violations found during the 3 experiment runs for each tuple of parameters.
Based on this figure, we can observe several interesting findings.\newline
Firstly, the amount and types of Accessibility Violations show a similar 
patterns across the different models. While overall \textit{gpt-4o} produced
slightly less violations, its numbers and distribution show a similar 
structure. Without refinement techniques, Color contrast violations, 
as well as HTML landmark and region 
issues are the predominantly cause for at least 75\% of all violations in both
gemini's flash-2.5 and in gpt-4o model and with diverse prompting techniques.\newline
While gemini seems to have more color contrast violations, 
landmark and region rules cause more problems for the gpt-4o model. Those 
findings are confirmed across different parameters.\newline
The subsequent types of violations are in the fields of missing labels,
links without distinguishable color, issues with the HTML header and 
the size of frontend components. The results show a noticeable trend that 
only a small subset of possible WCAG violations cause non-negligible amounts 
of violations in an Image-to-Code environment. Especially in the case of 
gpt-4o, there are only 6 types of violations which cause more than 10
violations combined across all prompting techniques in our dataset.\newline
Similar to related papers in this field (paper xxx), our results also show 
that different prompting techniques only have small impact on the findings.
Even if the naive prompting approach does not instruct the LLMs to generate
code with compliance to the WCAG standards, it still only shows slightly 
increased amounts of violations than more advanced prompting techniques such as 
zero-shot or reasoning.\newline
Figure xyy shows the violations found in the ground truth HTML/CSS of our 
dataset. The human-written HTML/CSS of our dataset counts 1339 accessibility 
violations, leading to $\sim$25.26 violations per file across the whole dataset.
On the other hand, even gemini with the naive prompting technique,
the worst performing set of parameter, had a maximum of 917 accessibility 
violations, leading to $\sim$17.3 violations per file across the whole dataset.
This shows that LLMs write by default more accessible code than humans by 
incorporating the most important rules. However, as described above they 
struggle with more complex rules such as color contrast or landmarks
that require further information or thinking.

IR noch mit aufnehmen.
IW-IR noch mit aufnehmen.
Verteilung der Issues bei Human und LLms mit aufnehmen.
Bar Charts noch mit aufnehmen ohne iterative


\subsection{Qualitive Analysis}
In a first step, we analyze how consistent violations are within different 
within different experiment runs and dataset entries. Figure xy shows the 
consistency of the type and the amount of a violation by using the 
\textit{cosine-similarity}. Therefore, we analyze the type of violation
and its amount per file in vector-like structure. 
The cosine-similarity is then calculated between the vectors of the 
different experiment runs and dataset entries. This allows us to 
analyze how similar the violations are and to understand how consistent
the LLMs are in generating accessible code.

\newcommand{\vect}[1]{\begin{pmatrix}#1\end{pmatrix}}
\newcommand{\issues}{k}                   % Anzahl der Issue-Klassen
\newcommand{\vx}{\mathbf x}
\newcommand{\vy}{\mathbf y}

\begin{enumerate}
  \item \textbf{Amount of Issues}  
        For each test run and file we are counting the number of violations per class
        \[
          \left(
            \begin{array}{@{}l r@{}}
              \text{Issue}_{1}: & x_{1}\\
              \text{Issue}_{2}: & x_{2}\\
              \vdots           & \vdots\\
              \text{Issue}_{\issues}: & x_{\issues}
            \end{array}
          \right)
          \xmapsto{\text{to vector}}
          \vx \;:=\;
          \vect{x_{1}\\x_{2}\\\vdots\\x_{\issues}} \in \mathbb R^{\issues}.
        \]

  \item \textbf{Calculation of Cosine Similarity}  
        Given two experiment runs with $\vx,\vy\in\mathbb R^{\issues}$, we define
        \[
          \operatorname{cos\_sim}(\vx,\vy)=
          \frac{\vx^{\mathsf T}\vy}{\lVert\vx\rVert\,\lVert\vy\rVert}
          \quad\in[0,1].
        \]
\end{enumerate}

This cosine similarity is then plotted into a heatmap comparing the different 
experiment runs and files. The results can be seen in figure ab below. It 
is noticeable that most of the tiles show a bright color, referring to a 
high cosine similarity. The tiles with a darker color are mainly caused 
by 2 reasons. The first reason are files with only little violations 
that cause smaller cosine similarities due to the non-consistent 
distribution of violations. The second reason are randomly-generated 
files which have been mutated in such a way that they chose colors that 
comply with the Color Contrast Rules. Since overall the color constrast 
issues are one of the most common issues, this leads to a lower cosine 
similarity.\newline
Those findings are consistent different models and 
prompting techniques. This demonstrates that the accessibility issues are 
consistent across multiple runs and not caused by halluzination of the 
LLMs but are based on their training data and the underlying parameters.\newline
In a last step, the question arises wether the violations found across
different dataset entries are caused by the same frontend components, such 
as a button or a label on the Input Image. This step required a mapping
of violations across the different results and their corresponding
components in the frontend. As table rt describes, $\sim$50\% of the 
violations in each run can be directly accounted to the same frontend 
elements. This demonstrates that half of the violations are caused by 
violations within the image inputs. Only the other half can not be directly 
connected to other violations and thus are caused by the LLMs' halluzinations.\newline
Especially, this qualitative analysis shows that pre-processing steps which 
analyze possible violations in the input image, could further reduce the 
amount of violations.


\section{Similarity Evaluation}
Since Image-to-Code main task is to copy the input image as precise as possible,
we have analysed the performance across the different parameter sets to see how 
exact their results remain. The results in table as in the appendix 
indicate that the final scores decrease slightly when further accessibility 
instructions are mentioned to the LLMs. While the text and position similarity 
remains constant, the size and especially the text color similarity scores 
decrease. This is caused due to accessibility compliance that can cause the
LLMs to choose different colors and even component sizes to align with the 
WCAG issues. However, the changes in terms of the final score are very small 
and almost negligible.\newline
Overall, similar to former research gpt-4o demonstrates the best performance 
in this field by outperforming gemini flash-2.0 by a few percent.
