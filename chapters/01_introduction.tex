% !TeX root = ../main.tex
% Add the above to each chapter to make compiling the PDF easier in some editors.

\chapter{Introduction}\label{chapter:introduction}

\section{Motivation}
High quality webpages are the backbone of our modern society. 
For billions of people the internet and thus webpages are the central access point 
for information, education, work, trade and culture. Irrespective of 
its content, such as shop, private blogs or news pages, each webpage is 
evaluated based on its usability.
However, yet the creation of webpages or user interfaces (UIs) follows a similar and repetitive 
pattern. First UI designs are created with the help of web-design tools. Those UI designs present 
the foundation for software developers. In a second step, they are translated into 
functional UI code which tries to resemble the intended layout and structure as 
close as possible. \newline
One essential, but yet frequently underestimated aspect of quality in this process is \textit{accessibility}:
According to the official \textit{WCAG} guidelines, code must be perceivable, operable, 
understandable and robust. This allows users with visual, hearing or cognitive
impairments, as well as users of assistive technologies, to follow the content
of a webpage.\newline
Complying with accessibility standards is not only an optional or moral aspect of 
web development, but it now has to follow regulatory boundaries. For instance the 
\textit{European Accessibility Act} came into 
effect on June 28, 2025 and obliges any e-commerce or digital service in the EU 
to comply with those standards. Neglecting to do so could result in warnings,
reputational damage and loss of sales in the future.\newline
At the same time, thanks to Large-Language Models (LLMs) we experience signifant 
improvements in automatic code generation. Current LLMs are capable to perform
\textit{Image-to-Code} tasks where based on a UI design or screenshot,
LLMs generate functional frontend code code. Recent research has demonstrated 
that especially for 
less complex webpages LLMs show decent performance~\parencite{si2024design2code}. \newline
But can LLMs also generate accessible frontend code in a realistic image-to-code scenario? 
This question has hardly been investigated to date. Especially due 
to rising frontend code generation, it will be interesting to see 
how the WCAG compliance will influence the visual similarity.

\subsection{Our Contributions}

\subsubsection{Evaluation Pipeline}
In order to close this gap, we propose a large scale accessibility evaluation pipeline of LLM-based
Image-to-Code generation. This pipeline combines visual and structural fidelity with an automatic 
WCAG conformity check. Therefore, we propose different benchmarks in 
order to measure the performance of the LLMs.

\subsubsection{Realistic Dataset}
We create a realistic dataset that contains of 53 real-world webpage 
examples which have been gathered from existing datasets and slightly mutated to
minimize noise within the data. It covers a wide spectrum of layouts, content 
areas and accessibility features. This dataset contains the screenshots 
of each webpage, but also the HTML/CSS created by human developers.

\subsubsection{Model Comparison}
We conduct an in-depth comparison of 3 state-of-the-art LLMs 
with vision capabilities (gpt-4o, gemini flash 2.0, qwen 7B vl) under the 
same experiment conditions. The LLMs are tested across different prompting 
strategies, as well as pre- and post-processing techniques.

\subsubsection{Quantitative and Qualitative Evaluation}
Each result will be analyzed based on a quantitative and qualitative 
evaluation. We analyze differences across the LLMs and outline their
reasons.

\subsubsection{Best-Practice Guidelines}
Based on our experience, we present concrete best-practice guidelines, how 
to combine different techniques in order to achieve a maximum amount 
of accessibility compliance.




