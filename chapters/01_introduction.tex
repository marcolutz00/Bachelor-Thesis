% !TeX root = ../main.tex
% Add the above to each chapter to make compiling the PDF easier in some editors.

\chapter{Introduction}\label{chapter:introduction}

\section{Section}
High quality web user interfaces are the backbone of our modern society. They allow us 
to present products or services in an interactive way and reach billions of users every day.
However, the creation of Websites or user interfaces (UIs) follows a similar and repetitive 
pattern. \newline
First UI designs are created with the help of special design tools. The UI designs present 
the foundation for software developers. In a second step, those designs are translated into 
functional UI code which tries to resemble the intended layout and structure, but also obey
to other design aspects. \newline
One essential, but yet frequently underestimated aspect in this process is \textit{accessibility}:
According to the official \textit{WCAG} guidelines, code must be  perceivable, operable, 
understandable and robust for people with various disabilities.\newline
Complying with accessibility standards is not only an optional, moral aspect of 
web development, but it now has to follow regulatory boundaries. Ensuring accessibility
is no longer optional. For instance the \textit{European Accessibility Act} comes into 
effect on June 28, 2025 and obliges any e-commerce or digital service in the EU 
to comply with those standards.\newline
Current Large-Language Models (LLMs) have shown signifant improvements in automatic
code generation. Especially \textit{Image-to-Code} tasks where UI designs are given as 
input and LLMs output the functional UI code, have been tested by various researchers in 
the past. Several benchmarks have shown the competitive performance of LLMs on those tasks.
However, the capability of modern LLMs to generate accessible Code in an Image-to-Code 
environment has only started to gain researchers interest quite recently. Existing 
research in this field have compared human to the generated code and tried to better
align with the accessibility standards. Nevertheless, this has never been tested in 
an Image-to-Code environment. Apart from that, accessibility does not only concern 
the visual appearance of a web user interace, but also its functionality. Thus, it 
requires the LLMs to have a deeper understanding than in classical Image-to-Code scenarios
where LLMs only reproduce the input images pixel perfectly.

\subsection{Our Contributions}
In order to close this gap, we propose a large scale accessibility evaluation of LLM-based
web code generation while also taking the visual similarity into account. Therefore,
we use a dataset which contains of 53 real-world webpage examples which have been 
gathered from existing datasets and mutated in order to prevent data leakage.
It covers a wide spectrum of layouts, content areas and accessibility features.\newline
This dataset






\subsection{Subsection}
Citation test~\parencite{latex}.
See~\autoref{tab:sample}, \autoref{fig:sample-drawing}, \autoref{fig:sample-plot}, \autoref{fig:sample-listing}.

\begin{table}[htpb]
  \caption[Example table]{An example for a simple table.}\label{tab:sample}
  \centering
  \begin{tabular}{l l l l}
    \toprule
      A & B & C & D \\
    \midrule
      1 & 2 & 1 & 2 \\
      2 & 3 & 2 & 3 \\
    \bottomrule
  \end{tabular}
\end{table}

\begin{figure}[htpb]
  \centering
  % This should probably go into a file in figures/
  \begin{tikzpicture}[node distance=3cm]
    \node (R0) {$R_1$};
    \node (R1) [right of=R0] {$R_2$};
    \node (R2) [below of=R1] {$R_4$};
    \node (R3) [below of=R0] {$R_3$};
    \node (R4) [right of=R1] {$R_5$};

    \path[every node]
      (R0) edge (R1)
      (R0) edge (R3)
      (R3) edge (R2)
      (R2) edge (R1)
      (R1) edge (R4);
  \end{tikzpicture}
  \caption[Example drawing]{An example for a simple drawing.}\label{fig:sample-drawing}
\end{figure}

\begin{figure}[htpb]
  \centering

  \pgfplotstableset{col sep=&, row sep=\\}
  % This should probably go into a file in data/
  \pgfplotstableread{
    a & b    \\
    1 & 1000 \\
    2 & 1500 \\
    3 & 1600 \\
  }\exampleA
  \pgfplotstableread{
    a & b    \\
    1 & 1200 \\
    2 & 800 \\
    3 & 1400 \\
  }\exampleB
  % This should probably go into a file in figures/
  \begin{tikzpicture}
    \begin{axis}[
        ymin=0,
        legend style={legend pos=south east},
        grid,
        thick,
        ylabel=Y,
        xlabel=X
      ]
      \addplot table[x=a, y=b]{\exampleA};
      \addlegendentry{Example A};
      \addplot table[x=a, y=b]{\exampleB};
      \addlegendentry{Example B};
    \end{axis}
  \end{tikzpicture}
  \caption[Example plot]{An example for a simple plot.}\label{fig:sample-plot}
\end{figure}

\begin{figure}[htpb]
  \centering
  \begin{tabular}{c}
  \begin{lstlisting}[language=SQL]
    SELECT * FROM tbl WHERE tbl.str = "str"
  \end{lstlisting}
  \end{tabular}
  \caption[Example listing]{An example for a source code listing.}\label{fig:sample-listing}
\end{figure}
