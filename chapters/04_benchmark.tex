\chapter{Benchmarks}\label{chapter:Benchmarks}

\section{Visual and Structural}
As the main instruction for LLMs for this study remains an Image-to-Code task, it 
is necessary to evaluate the generated HTML/CSS code based on visual and structural
similarity. One approach which seems very promising has been presented in the 
paper Design2Code. This approach is more fine-grained than former ideas, since 
it compares the input and the output on a component-level rather than in its entirety.
The authors describe a sophisticated matching algorithm that combines the 
HTML elements in the ground truth with those in the generated code. Based on this 
matching, it is then possible to run several metrics, such as text-similarity,
position-similarity, color-difference, clip score and area sum score.

\section{Accessibility}
In order to measure and compare the accessibility improvements in terms of 
quantity and severity of the violations we use two different metrics.\newline
The first metric is the \textit{Inaccessibility Rate}(IR) which has been used in previous researchs 
in this field. This metric divides the amount of nodes with accessibility violations 
by the amount of nodes which are susceptible for violations. The interpretation of 
this metric is straight-forward, since it allows us to get the percentage of nodes
with violations compared to the total amount of nodes.\newline
\begin{equation}
    \mathrm{IR} = \frac{N_{\mathrm{violations}}}{N_{\mathrm{total}}}
\end{equation}
However, the Inaccessibility Rate does not take the severity of issues into account.
Thus, we have created the \textit{Impact-Weighted Inaccessibility Rate}(IWIR). This metric 
uses the impacts of Accessibility Violations found (minor, moderate, serious, critical)
and assigns them to a value (1, 3, 6, 10). Our scoring reflects the non-linear 
increase in impact for people with disablities if a violation with a higher 
impact takes place within the code. Finally, the Impact-Weighted Inaccessibility Rate 
is calculated by creating the sum over all Violations found multiplied by the 
corresponding value for the severity. This sum is then divided by the amount of violations
found multiplied by the highest impact score. By dividing the sum above through the worst 
possible outcome, a situation where every violation is critical, allows to get an 
understanding of the severity of the violations found.\newline
\begin{equation}
  \mathrm{IWIR} = 
    \frac{\displaystyle\sum_{i=1}^{k} v_i \, w_i}
         {\displaystyle\sum_{i=1}^{k} v_i \, w_{\mathrm{max}}}
  \label{eq:iwir}
\end{equation}
The combination of both metrics allows us to understand wether LLMs can not only decrease
the amount of accessibility violations, but also its severity.


