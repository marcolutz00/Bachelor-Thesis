\chapter{Appendix}\label{chapter:Appendix}

\section{Prompts}
\definecolor{myblue}{RGB}{0,55,205}

\tikzset{
  promptbox/.style={
    draw=myblue,
    dashed,
    dash pattern=on 3pt off 2pt,
    rounded corners=6pt,
    line width=.8pt,
    font=\normalsize,
    inner sep=6pt,
    align=left,
  },
}

\newlength{\hgap}\setlength{\hgap}{6mm}   % horizontaler Abstand
\newlength{\vgap}\setlength{\vgap}{6mm}   % vertikaler Abstand

% Breite einer „normalen“ Box   (3 Stück + 2 Lücken = \textwidth)
\newlength{\boxwidth}
\setlength{\boxwidth}{\dimexpr(\textwidth-2\hgap)/3\relax}

% Breite der großen Box (nimmt zwei Spalten ein)
\newlength{\bigboxwidth}
\setlength{\bigboxwidth}{\dimexpr2\boxwidth+\hgap\relax}



\begin{figure}[ht]
\centering
\begin{tikzpicture}[node distance=\hgap]  % stellt den h-Abstand
%--------------------------------------------------------------
%  REIHE 1 – zwei Boxen
%--------------------------------------------------------------
% 1 a) Direct Prompt (linke Box oben)
\node[promptbox,
      text width=\bigboxwidth,
      anchor=north west] (direct)
{%
  \centering\bfseries Naive Prompt\\[4pt]
  \textcolor{myblue}{\emph{(Instruction):}}\\
  {\footnotesize 
  You are an expert web developer who specializes in HTML/CSS.
  Your task is to replicate the provided UI screenshot of a webpage pixel-perfectly using HTML/CSS.\newline
  \par}

  \textcolor{myblue}{\emph{[Guidelines]:}}\\
  \begin{enumerate}\itemsep2pt
    {\footnotesize \item \textbf{Layout}: Structure the markup so the spatial arrangement of every element exactly matches the screenshot.\par}
    {\footnotesize \item \textbf{Styling}: Reproduce fonts, colors, spacing, sizes, borders, shadows, and any other visual details as closely as possible.\par}
    {\footnotesize \item \textbf{Content}: Include all visible text, icons, and graphic elements.\par}
    {\footnotesize \item \textbf{Image placeholders}: Blue boxes represent images. Use `<img>` and as a source the placeholder file `src="src/placeholder.jpg"`
    to reserve space. However, the appropriate height and width is very important.\par}
    {\footnotesize \item \textbf{Delivery format}: Output only the complete HTML and CSS code in one file — no additional comments or explanations.\par}
  \end{enumerate}

   {\footnotesize  
    Now convert the following image into HTML/CSS according to these requirements.
    \par}
};

% 1 b) zweite Box oben (Beispiel: Mark Prompt)
\node[promptbox,
      text width=\boxwidth,
      right=\hgap of direct.north east,
      anchor=north west] (markTop)
{%
  \centering\bfseries Accessibility Reminder\\[4pt]
  \textcolor{myblue}{\emph{(Instruction):}}\\
  {\footnotesize It is EXTREMELY important that your HTML/CSS complies with WCAG 2.2.\\[2pt]
  Refer to the full spec if in doubt:\\ \par} 
  \href{https://www.w3.org/TR/WCAG22/}{\underline{\texttt{Link to WCAG 2.2}}}\\[4pt]
  \centering {\footnotesize Avoid any violations. \par}
};

%--------------------------------------------------------------
%  Koordinate für die zweite Reihe
%--------------------------------------------------------------
\coordinate (row2) at ($(direct.south west)+(0,-\vgap)$);

%--------------------------------------------------------------
%  REIHE 2 – drei Boxen
%--------------------------------------------------------------
% 2 a) Chain-of-Thought Prompt
\node[promptbox,
      text width=\boxwidth,
      anchor=north west] (cot) at (row2)
{%
  \centering\bfseries Zero-Shot Prompt\\[4pt]
  \textcolor{myblue}{\emph{(Instruction):}} \\[2pt]
  {\underline{\texttt{Naive Prompt}}}\\+\\
  {\underline{\texttt{Accessibility Reminder}}}
};

% 2 b) Failure-aware Prompt
\node[promptbox,
      text width=\boxwidth,
      right=\hgap of cot] (fap)
{%
  \centering\bfseries Few-Shot Prompt\\[4pt]
  \textcolor{myblue}{\emph{(Instruction):}}
  Blublub%
};

% 2 c) dritte Box unten  – Platzhalter
\node[promptbox,
      text width=\boxwidth,
      right=\hgap of fap] (extra)
{%
  \centering\bfseries Chain-of-Thought Prompt (CoT)\\[4pt]
  \textcolor{myblue}{\emph{(Instruction):}} \\[2pt]
  {\underline{\texttt{Naive Prompt}}}\\+\\
  {\underline{\texttt{Accessibility Reminder}}}\\+\\
  \centering {\footnotesize Let’s think step by step.\newline\newline
  \underline{Output Format} \\
  When you reply:\\
  1. Return ONE valid JSON object and nothing else.\\
  2. It MUST have exactly these keys:\\
  • "thoughts" – your internal reasoning\\
  • "code" – a complete HTML/CSS document as described above \par}
};

\end{tikzpicture}
\caption{Overview of Prompts.}
\end{figure}