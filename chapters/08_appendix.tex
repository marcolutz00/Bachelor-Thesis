\chapter{Appendix}\label{chapter:Appendix}

\section{Prompts}
\definecolor{myblue}{RGB}{0,55,205}

\tikzset{
  promptbox/.style={
    draw=myblue,
    dashed,
    dash pattern=on 3pt off 2pt,
    rounded corners=6pt,
    line width=.8pt,
    font=\normalsize,
    inner sep=6pt,
    align=left,
  },
}

% Breiten für die beiden Reihen (nach Belieben anpassen)
\newcommand{\topboxwidth}{0.47\textwidth}
\newcommand{\botboxwidth}{0.30\textwidth}
\newcommand{\hgap}{6mm}   % horizontaler Abstand zwischen Boxen
\newcommand{\vgap}{6mm}

\begin{figure}[ht]
\centering
\begin{tikzpicture}[node distance=\hgap]  % stellt den h-Abstand
%--------------------------------------------------------------
%  REIHE 1 – zwei Boxen
%--------------------------------------------------------------
% 1 a) Direct Prompt (linke Box oben)
\node[promptbox,
      text width=\topboxwidth,
      anchor=north west]                    % links oben beginnen
      (direct)
{%
  \centering\bfseries Naive Prompt\\[4pt]
  \textcolor{myblue}{\emph{(Instruction):}}\\
  {\footnotesize 
  You are an expert web developer who specializes in HTML and CSS.
  Your task is to replicate the provided UI screenshot of a webpage pixel-perfectly using HTML and CSS.\newline
  \par}

  \textcolor{myblue}{\emph{[Guidelines]:}}\\
  \begin{enumerate}\itemsep2pt
    \item {\footnotesize \textbf{Layout}: Structure the markup so the spatial arrangement of every element exactly matches the screenshot.\par}
    \item {\footnotesize \textbf{Styling}: Reproduce fonts, colors, spacing, sizes, borders, shadows, and any other visual details as closely as possible.\par}
    \item {\footnotesize \textbf{Content}: Include all visible text, icons, and graphic elements.\par}
    \item {\footnotesize \textbf{Image placeholders}: Blue boxes represent images. Use `<img>` and as a source the placeholder file `src="src/placeholder.jpg"`
    to reserve space. However, the appropriate height and width is very important.\par}
    \item {\footnotesize \textbf{Delivery format}: Output **only** the complete HTML and CSS code in **one file** — no additional comments or explanations.\par}
  \end{enumerate}

   {\footnotesize  
    Now convert the following image into HTML/CSS according to these requirements.
    \par}
};

% 1 b) zweite Box oben (Beispiel: Mark Prompt)
\node[promptbox,
      text width=\topboxwidth,
      right=\hgap of direct.north east,     % rechts daneben
      anchor=north west]                    % obere Kante ausrichten
      (markTop)
{%
  \centering\bfseries Accessibility Reminder\\[4pt]
  \textcolor{myblue}{\emph{(Instruction):}}
  In the first screenshot, elements are highlighted…%
};

%--------------------------------------------------------------
%  Koordinate für die zweite Reihe
%--------------------------------------------------------------
\coordinate (row2) at ($(direct.south west) + (0,-\vgap)$); % direkt unter Direct Prompt

%--------------------------------------------------------------
%  REIHE 2 – drei Boxen
%--------------------------------------------------------------
% 2 a) Chain-of-Thought Prompt
\node[promptbox,
      text width=\botboxwidth,
      anchor=north west] (cot) at (row2)
{%
  \centering\bfseries Zero-Shot Prompt\\[4pt]
  \textcolor{myblue}{\emph{(sg):}} You should think step by step:\\[2pt]
  \begin{enumerate}\itemsep0pt
    \item Understand the interaction…
    \item Locate interactive elements…
    \item Implement the interaction…
  \end{enumerate}
  Combine HTML, CSS and JavaScript codes into one file…%
};

% 2 b) Failure-aware Prompt
\node[promptbox,
      text width=\botboxwidth,
      right=\hgap of cot] (fap)
{%
  \centering\bfseries Few-Shot Prompt (FAP)\\[4pt]
  \textcolor{myblue}{\emph{(Instruction):}}
  There are ten types of errors you should avoid…%
};

% 2 c) dritte Box unten  – Platzhalter
\node[promptbox,
      text width=\botboxwidth,
      right=\hgap of fap] (extra)
{%
  \centering\bfseries Chain-of-Thought Prompt (CoT)\\[4pt]
  Hier kannst du bei Bedarf eine fünfte Bubble ergänzen…%
};

\end{tikzpicture}
\caption{Overview of Prompts.}
\end{figure}