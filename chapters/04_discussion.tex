\chapter{Discussion}\label{chapter:discussion}
This section discusses the findings of this thesis 
and explores possible implications for the future 
of accessibility-aware code generation.

\section{Rethinking Accessibility as a First-Class Objective in Code Generation}
Even though state-of-the-art MLLMs have shown impressive 
capabilities in generating UI images into syntactically 
valid HTML/CSS code, accessibility remains a significant 
challenge. When prompted natively, even the most advanced models, 
GPT-4o and Gemini-2.0 flash, show only little awareness 
of accessibility best practices.
These models often generate code that is similar to the 
original UI image but consistently fails to meet 
the accessibility standards set by the WCAG.
This highlights the necessity of a paradigm 
shift in the approach of code generation. 
In practice, the conclusions of this thesis 
often reveal that improving 
the overall model quality is not sufficient to guarantee
accessibility. 
Functional and visual correctness of the generated code 
can be orthogonal, or even in conflict with accessibility. 
A layout that renders correctly, might still not be navigable 
by keyboard users or usable for screen reader users.\newline 
Therefore, these findings highlight the need to rethink 
accessibility as a first-class objective in model
development, prompting, and evaluation. This shift 
means more than just incorporating accessibility 
checks as a post-processing step, but rather
integrating it proactively into the design principles 
of the models. For instance, this would include that 
accessibility metrics should be reported alongside traditional 
code quality metrics, especially in domains 
where accessibility is crucial, such as web development.\newline 
To get insights from developers working on 
a daily basis with LLMs, a small survey and discussions~\cite{feng2025ux, lutz2025mot} 
has been conducted in a couple of developer communities.
The feedback from these discussions and some private interviews 
highlights how 
LLMs are perceived in practice: \textit{``LLMs can point 
you in the right direction, but they struggle with specific fixes''}.
Others cautioned that \textit{``over-reliance without fact-checking will 
ultimately worsen the experience for the user''}. This 
feedback underlines an important point: while LLMs can 
assist and accelerate the development process, they can not yet
be trusted to ensure accessibility without human oversight. 
This highlights the necessity for accessibility to be 
an optimization goal in the entire code generation process.


\section{From UI Description to Code: An Even Greater Accessibility Challenge}
While the scope of this thesis focused on the evaluation of 
generated code from UI screenshots, recent advancements could 
shift the focus from \emph{image-based generation} to 
\emph{description-based generation}. In this context, 
the model is prompted via natural language or structured 
descriptions and generates code directly from these inputs.
However, this shift introduces even greater challenges 
and a greater risk of omission of accessibility features.\newline
Unlike with screenshots, where the model can visually get 
information about layout and stylistic details, text 
descriptions may not mention accessibility requirements at all.
The responsibility of including accessibility information falls entirely 
on the developer, who must make sure that the description 
contains all necessary details. However, this includes the 
assumption that the developer is aware of the accessibility requirements, 
which is often not the case. For instance, a developer
might write \textit{``Create a login form with username 
and password fields''}, without mentioning that specific 
labels, keyboard focus orders, or ARIA roles are required.
Additionally, accessibility-related information, such as
the choice of colors, alt-texts for images, or keyboard 
navigation details, can be dependent on the context and 
subjective for each individual developer. Understanding these 
details from language alone is a significant challenge for
LLMs. For example, a model might not be capable of determining
which elements on a webpage should be reachable via keyboard
or if the color contrast is sufficient for the user. \newline
Therefore, the shift from image-based to description-based
code generation poses even more challenges for accessibility and 
demonstrates the need for explicit alignment and prompt design 
to ensure that accessibility considerations are adequately addressed.
