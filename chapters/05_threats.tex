 \chapter{Threats to Validity}\label{chapter:threats}
This chapter introduces possible threats to the validity of 
the thesis and demonstrates how these threats 
are mitigated.\newline
First, the representativeness of the UI dataset poses 
a potential threat to the generalizability of the findings.
In order to reduce this threat, the data entries are 
collected from two publicly available datasets (Design2Code and WebCode2m).
These two datasets are manually inspected and found to contain a diverse 
range of UI designs, of which 53 examples are selected for the study. 
Those examples cover a wide range of UI elements, design, and 
content types. Additionally, a synthetic dataset, consisting of 
20 entries, is created to 
further increase the diversity and rule out the risk of 
potential data leakage.\newline 
Second, the choice of models poses another potential threat 
to this thesis. To address this, the study includes four widely-used 
and state-of-the-art MLLMs, such as GPT-4o, Gemini 2.0 Flash, Qwen2.5-VL-7B,
and LLaVA-7B. The thesis has intentionally not included 
task-optimized or fine-tuned models to reflect realistic 
usage scenarios. Apart from that, possible resource constraints 
are also considered. The thesis evaluates the open-source 
models at the scale of 7B parameters, rather than using their 
large variants (e.g., Qwen-2.5-VL-72B), which might show 
a stronger performance, but are less accessible to most users.
This selection of models captures the diversity of MLLMs at 
the current state, but also acknowledges that further 
model families and variants could enhance the representativeness of the findings.\newline
Third, the models' prompting strategies pose another potential threat 
to the validity of the findings. To reduce the risk of 
prompting bias, the thesis has evaluated the performance of 
the models with a set of 7 different prompting techniques, 
ranging from a naive prompt without any additional accessibility instructions
to more advanced prompts that include external accessibility
checkers and guidelines. While there are many more techniques 
that have shown promising results in the literature, such as 
\textit{prompt tuning} or \textit{model context protocol} (MCP), 
the resource constraints of this thesis limit the number of
techniques that could be evaluated. The prompt design used 
in this study, however, was intended to be representative of 
general use and aligned with best practices of prior research.\newline 
Lastly, another threat to validity is the probabilistic and 
non-deterministic nature of MLLMs, which can lead to 
variations in the observed results, even for the same input.
To mitigate this threat, the 
study has evaluated the performance of each model 
based on three different runs and reported the aggregated, 
averaged results. While this approach helps to reduce the 
number of possible outliers, it does not fully eliminate 
them. Overall, the observations in this thesis showed variations 
between the runs, but the overall trends and findings 
remained consistent. Nevertheless, a larger number of 
runs could further improve the robustness of the findings.